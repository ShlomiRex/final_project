\documentclass{article}
\usepackage{amsmath}
\usepackage{amsfonts}
\usepackage{geometry}
\usepackage{graphicx}
\usepackage{hyperref}

\geometry{margin=1in}

\author{}
\date{}

\begin{document}


The forward diffusion process is defined by:

\begin{equation}
q(x_t \mid x_{t-1}) = \mathcal{N}(x_t; \sqrt{\alpha_t} x_{t-1}, (1 - \alpha_t) \mathbf{I})
\end{equation}

where $\alpha_t \in (0, 1)$ controls the amount of signal retained at each step, and $(1 - \alpha_t)$ determines the variance of the added Gaussian noise. Define $\beta_t = 1 - \alpha_t$ for convenience.

\section*{Recursive Substitution}

By repeatedly applying the forward process from $x_0$ to $x_t$, we can marginalize out all intermediate steps. \textbf{The key insight from the DDPM paper is that the marginal distribution of $x_t$ given $x_0$ is also Gaussian:}

\begin{equation}
q(x_t \mid x_0) = \mathcal{N}(x_t; \sqrt{\bar{\alpha}_t} x_0, (1 - \bar{\alpha}_t) \mathbf{I})
\end{equation}

where $\bar{\alpha}_t = \prod_{s=1}^t \alpha_s$ (basically we run all the steps from $1$ to $t$, which has a Gaussian marginal distribution).

Since the marginal distribution is known, we can sample $x_t$ directly from $x_0$ using:

\begin{equation}
x_t = \sqrt{\bar{\alpha}_t} x_0 + \sqrt{1 - \bar{\alpha}_t} \, \epsilon, \quad \epsilon \sim \mathcal{N}(0, \mathbf{I})
\end{equation}

This is the equation used during training to generate noisy images $x_t$ from clean images $x_0$ in a single step, making the process efficient.


\end{document}

\documentclass{article}
\usepackage{graphicx}
\usepackage{amsmath}
\usepackage{amsfonts}
\usepackage{hyperref}

\title{Supplementary Material}
\author{}
\date{}

\begin{document}

\maketitle

\section{Additional Data}
Here you can include any additional data that supports your findings, such as extended tables, figures, or datasets.

\section{Extended Methodology}
Provide further details on the methodology that were not included in the main text. This can include additional equations, algorithms, or experimental setups.

\section{Additional Figures}
\begin{figure}[h]
    \centering
    \includegraphics[width=0.8\textwidth]{figures/example_figure.png}
    \caption{Example of an additional figure that supports the findings.}
    \label{fig:example}
\end{figure}

\section{Additional Tables}
\begin{table}[h]
    \centering
    \begin{tabular}{|c|c|c|}
        \hline
        Parameter & Value & Description \\
        \hline
        Param 1 & 10 & Description of Param 1 \\
        Param 2 & 20 & Description of Param 2 \\
        \hline
    \end{tabular}
    \caption{Additional parameters used in the experiments.}
    \label{tab:additional_parameters}
\end{table}

\section{References}
Include any additional references that support the supplementary material.

\end{document}
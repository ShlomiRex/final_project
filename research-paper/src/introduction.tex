\section{Introduction}

Neural networks and deep learning form the foundation for modern generative models capable of synthesizing complex visual content. Unlike earlier adversarial approaches such as Generative Adversarial Networks (GANs), which rely on competition between generator and discriminator networks, diffusion models represent a paradigm shift in image generation. Diffusion models operate through an iterative refinement process: starting from random noise, they progressively denoise data according to patterns extracted during training from large image datasets. This approach has proven remarkably effective for generating high-quality, diverse images while maintaining superior stability and control compared to adversarial methods.

Stable Diffusion is a state-of-the-art diffusion-based generative model that synthesizes images through iterative noise reduction (see Figure~\ref{fig:ddpm_noise}). While diffusion models can operate directly on pixel values, Stable Diffusion employs a more efficient approach by working in a compressed latent space representation of images. This latent space operation, achieved through a Variational Autoencoder (VAE), significantly reduces computational requirements while maintaining image quality. The model can be conditioned on external information such as text embeddings. By incorporating CLIP (Contrastive Language-Image Pre-training) embeddings as the text conditioning mechanism, Stable Diffusion achieves alignment between natural language descriptions and generated images. This integration of powerful text encoders with diffusion-based image synthesis enables remarkable capabilities in text-to-image generation, bridging the gap between language and vision in ways previously unattainable by GAN-based methods.

\begin{figure}[h]
    \centering
    \includegraphics[width=0.8\textwidth]{../figures/ddpm_noise.png}
    \caption{The iterative denoising process in diffusion models: starting from random noise, the model progressively refines the image through multiple steps.}
    \label{fig:ddpm_noise}
\end{figure}

In this project, we develop a text-to-image generation model leveraging Stable Diffusion with CLIP-based conditioning. Our implementation demonstrates substantial performance improvements over existing GAN-based approaches. The superior performance of diffusion models stems from their inherent stability, superior sample quality, and greater flexibility in conditioning mechanisms. We explore these advantages through comprehensive experiments using established evaluation metrics and comparative analysis, establishing diffusion models as a superior alternative to adversarial methods for text-conditioned image synthesis.

\textbf{[TODO: Revise the following paragraph]}

The paper is structured as follows: Section~\ref{sec:related_work} reviews related work and existing approaches in generative modeling. Section~\ref{sec:methodology} provides a detailed description of our Stable Diffusion architecture, CLIP conditioning mechanism, and training approach. Section~\ref{sec:experiments} presents our experimental setup and training protocol. Section~\ref{sec:results} presents the results of our experiments and comparative analysis with GAN-based methods. Finally, Section~\ref{sec:discussion} discusses the implications of our findings, the advantages of diffusion-based models, and directions for future work.
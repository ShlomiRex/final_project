\section{Experimental Setup}

\subsection{Hardware and Software Configuration}

\begin{itemize}
    \item \textbf{Hardware:} HPC cluster with GPU nodes (specifications vary by node type)
    \item \textbf{CUDA:} Version 11.8
    \item \textbf{Python:} 3.10
    \item \textbf{Deep Learning Framework:} PyTorch 2.7.1+cu118
    \item \textbf{Key Libraries:}
    \begin{itemize}
        \item HuggingFace Diffusers (diffusion models and schedulers)
        \item HuggingFace Transformers (CLIP text encoder)
        \item torchvision (dataset loading)
        \item matplotlib (visualization)
    \end{itemize}
\end{itemize}

\subsection{Dataset}

\textbf{MNIST Handwritten Digits Dataset:}
\begin{itemize}
    \item \textbf{Size:} 60,000 training images, 10,000 test images
    \item \textbf{Resolution:} 28×28 pixels, grayscale
    \item \textbf{Classes:} 10 digit classes (0-9)
    \item \textbf{Text captions:} Automatically generated as "A handwritten digit \{label\}"
    \item \textbf{Preprocessing:} Conversion to tensors with values in range [0, 1]
\end{itemize}

\section{Experiment 1: Baseline Training}

\subsection{Objective}
Train a text-conditioned diffusion model to generate handwritten digits from text prompts, establishing baseline performance.

\subsection{Training Configuration}
\begin{itemize}
    \item \textbf{Initial training:} 1 epoch for validation
    \item \textbf{Extended training:} 5 epochs
    \item \textbf{Batch size:} 512
    \item \textbf{Learning rate:} $10^{-3}$
    \item \textbf{Optimizer:} AdamW
    \item \textbf{Number of trainable parameters:} Approximately 2.6M (UNet only)
\end{itemize}

\subsection{Monitoring}
Training progress monitored by:
\begin{itemize}
    \item Loss tracking every 25 steps
    \item Final epoch loss reporting
    \item Visual inspection of generated samples
\end{itemize}

\section{Experiment 2: Classifier-Free Guidance Ablation}

\subsection{Objective}
Investigate the impact of guidance scale on generation quality and text-image alignment.

\subsection{Guidance Scale Values}
We systematically evaluate the following guidance scales:
\begin{itemize}
    \item $w = 0$ (unconditional generation, no text guidance)
    \item $w = 5$ (weak guidance)
    \item $w = 8$ (moderate guidance, baseline)
    \item $w = 10$ (strong guidance)
    \item $w = 20$ (very strong guidance)
    \item $w = 50$ (extreme guidance)
    \item $w = 100$ (maximum guidance)
\end{itemize}

\subsection{Evaluation Protocol}
For each guidance scale:
\begin{enumerate}
    \item Generate images with fixed prompt: "A handwritten digit 0"
    \item Use fixed random seed (422) for reproducibility
    \item Apply 50 denoising steps
    \item Qualitatively assess:
    \begin{itemize}
        \item Image clarity and sharpness
        \item Adherence to text prompt
        \item Presence of artifacts or distortions
        \item Overall sample quality
    \end{itemize}
\end{enumerate}

\subsection{Inference Parameters}
\begin{itemize}
    \item \textbf{Scheduler:} DDPM with squared cosine schedule
    \item \textbf{Number of inference steps:} 50
    \item \textbf{Random seed:} 422 (for reproducibility)
    \item \textbf{Batch size:} 1 image per generation
\end{itemize}

\section{Experimental Procedure}

\subsection{Training Workflow}
\begin{enumerate}
    \item Load pretrained CLIP text encoder and freeze weights
    \item Initialize custom UNet with reduced parameters
    \item Create MNIST dataset with automatic caption generation
    \item Train for specified number of epochs with progress logging
    \item Save model checkpoints (optional)
\end{enumerate}

\subsection{Generation Workflow}
\begin{enumerate}
    \item Set models to evaluation mode
    \item Tokenize text prompt using CLIP tokenizer
    \item Encode text and empty string for CFG
    \item Initialize random noise tensor
    \item Iteratively denoise using scheduler and UNet predictions
    \item Post-process output (normalize to [0, 255] for visualization)
\end{enumerate}

\subsection{Visualization}
Generated images displayed using matplotlib with:
\begin{itemize}
    \item Grayscale colormap
    \item Grid layout for guidance scale comparison
    \item Individual titles showing guidance scale values
    \item Tight layout to prevent subplot overlap
\end{itemize}


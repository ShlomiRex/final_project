\section{Experiments}

\subsection{Experimental Setup}

\subsubsection{Hardware Environment}

\begin{itemize}
    \item \textbf{Environment:} HPC cluster with GPU nodes (specifications vary by node type)
    \item \textbf{Cluster resources (node available):}
    \begin{itemize}
        \item \textbf{CPU}: Intel Xeon Gold 6330 (56 cores @ 2.00 GHz)
        \item \textbf{GPUs}: 8x NVIDIA A100 80GB PCIe (80GB VRAM)
        \item \textbf{Memory (RAM)}: Node total 256 GB
    \end{itemize}
    \item \textbf{Requested resources (SLURM allocation used for experiments):}
    \begin{itemize}
        \item \textbf{CPUs}: 4 logical cores requested via `--cpus-per-task=4` (experiments used 4 cores)
        \item \textbf{Memory}: 32 GB requested via `--mem=32G`
        \item \textbf{GPUs}: 1 GPU (NVIDIA A100) requested via `--gres=gpu:1`
    \end{itemize}
\end{itemize}

\subsubsection{Software Environment}

\begin{itemize}
    \item \textbf{CUDA:} Version 11.8
    \item \textbf{Python:} 3.10.19 (in different experiments I used 3.11 as well)
    \item \textbf{Deep Learning Framework:} PyTorch 2.7.1+cu118
\end{itemize}

\subsubsection{Distributed Training Considerations}

Although the HPC environment supports multi-GPU and multi-node jobs, distributed training via SLURM proved unreliable in practice due to recurring configuration and environment issues. Additionally, distributed training with Python scripts requires external experiment tracking tools (e.g., MLflow) to monitor training progress, involving significant setup overhead: configuring image logging, artifact storage locations, and experiment tracking infrastructure. In contrast, Jupyter notebooks provide immediate visual feedback on training progress, loss curves, and generated samples without additional tooling. To keep the study focused and reproducible with minimal overhead, all experiments were conducted on a single NVIDIA A100 80GB GPU using Jupyter notebooks. Future work may revisit distributed training using a non-interactive script and a unified environment submitted as SLURM batch jobs, with proper experiment tracking infrastructure in place.

\subsection{MNIST Text-to-Image Generation with Classifier-Free Guidance}

\begin{figure}[h]
    \centering
    \includegraphics[width=\textwidth]{../figures/experiments/mnist_prompts.png}
    \caption{Generated images for prompt "A handwritten digit 3" across different guidance scales. Higher guidance scales produce sharper, more confident outputs with stronger adherence to the text prompt.}
    \label{fig:mnist_prompts}
\end{figure}

\subsubsection{Objective}
This experiment tests the fundamental principles of text-to-image generation using a diffusion-based architecture. By training on MNIST handwritten digits as a minimal-scale dataset, we investigate how the stable diffusion model handles image synthesis from text prompts and systematically evaluate the impact of Classifier-Free Guidance (CFG) on generation quality and text-image alignment.

\subsubsection{Model Architecture}

\textbf{Text Encoder (Frozen):}
\begin{itemize}
    \item \textbf{Model:} CLIP (openai/clip-vit-base-patch32)
    \item \textbf{Embedding dimension:} 512
    \item \textbf{Tokenizer max length:} 8 tokens (reduced from default 77)\footnote{Prompts are tokenized then padded or truncated to exactly 8 tokens. This keeps shapes fixed (batch, 8, 512) for CLIP embeddings and reduces unnecessary padding/compute versus the 77-token default. Longer prompts would be clipped beyond 8 tokens, which is acceptable here because prompts are intentionally short (e.g., "A handwritten digit 5").}
    \item \textbf{Training:} Weights are frozen
\end{itemize}

\textbf{Denoising Network (U-Net):}

The key design choice in the U-Net architecture is to train directly in pixel space without a VAE, which is viable for MNIST's low resolution (28$\times$28).

\begin{itemize}
    \item \textbf{Architecture:} Custom UNet2DConditionModel
    \item \textbf{Input/Output:} 1 channel (grayscale), 28$\times$28 pixels
    \item \textbf{Block channels:} (32, 64, 64, 32)
    \item \textbf{Layers per block:} 2
    \item \textbf{Down blocks:} DownBlock2D → CrossAttnDownBlock2D → CrossAttnDownBlock2D → DownBlock2D
    \item \textbf{Up blocks:} UpBlock2D → CrossAttnUpBlock2D → CrossAttnUpBlock2D → UpBlock2D
    \item \textbf{Cross-attention dimension:} 512 (matches CLIP embedding size)
    \item \textbf{Total trainable parameters:} 3,140,385
\end{itemize}

\subsubsection{Dataset}

\textbf{MNIST Handwritten Digits:}
\begin{itemize}
    \item \textbf{Training images:} 60,000
    \item \textbf{Resolution:} 28$\times$28 pixels, grayscale
    \item \textbf{Classes:} 10 digit classes (0-9)
    \item \textbf{Text captions:} Automatically generated as "A handwritten digit \{label\}" (e.g., "A handwritten digit 5")
    \item \textbf{Preprocessing:} Conversion to tensors with values normalized to [0, 1]
\end{itemize}

\begin{figure}[h]
    \centering
    \includegraphics[width=0.8\textwidth]{../figures/experiments/mnist_training_dataset.png}
    \caption{Sample images from MNIST training dataset showing handwritten digits (0-9) with corresponding class labels.}
    \label{fig:mnist_training_dataset}
\end{figure}

\subsubsection{Training Configuration}

\begin{itemize}
    \item \textbf{Batch size:} 512
    \item \textbf{Learning rate:} $10^{-3}$
    \item \textbf{Optimizer:} AdamW
    \item \textbf{Epochs:} 1 (initial validation), then 5 (extended training), and finally 20 epochs for final model
    \item \textbf{Noise scheduler:} DDPM with squared cosine beta schedule
    \item \textbf{Timesteps:} 1,000
    \item \textbf{Loss function:} Mean Squared Error (MSE) between predicted and actual noise
\end{itemize}

\textbf{Training Pipeline per Batch:}
\begin{enumerate}
    \item Convert digit labels to text captions using CLIP tokenizer
    \item Encode captions to semantic embeddings via frozen CLIP text encoder [batch, 8, 512]
    \item Sample random timestep $t \sim \text{Uniform}(0, 1000)$ for each image
    \item Add Gaussian noise to images: $x_t = \sqrt{\bar{\alpha}_t} x_0 + \sqrt{1-\bar{\alpha}_t} \epsilon$
    \item Predict noise: $\epsilon_\theta(x_t, t, c_\text{text})$ using UNet with cross-attention conditioning
    \item Calculate MSE loss: $\mathcal{L} = \|\epsilon - \epsilon_\theta(x_t, t, c_\text{text})\|^2$
    \item Backpropagate and update UNet parameters only (CLIP remains frozen)
\end{enumerate}

\textbf{Monitoring:}
\begin{itemize}
    \item Loss tracking every 25 steps
    \item Final epoch loss reporting
    \item Visual inspection of generated samples
\end{itemize}

\subsubsection{Inference and Classifier-Free Guidance}

\textbf{Basic Sampling (Without CFG):}
\begin{itemize}
    \item Initialize random noise tensor: $x_T \sim \mathcal{N}(0, I)$
    \item Encode text prompt via CLIP
    \item Iteratively denoise for 50 steps using DDPM scheduler
    \item Post-process: normalize output to [0, 255] for visualization
\end{itemize}

\textbf{Classifier-Free Guidance (CFG):}

CFG enables stronger text conditioning by computing both conditional and unconditional predictions:

\begin{enumerate}
    \item \textbf{Dual embeddings:}
    \begin{itemize}
        \item Conditional: Text prompt → CLIP embedding $c_\text{text}$
        \item Unconditional: Empty string "" → CLIP embedding $c_\text{null}$
    \end{itemize}
    \item \textbf{Guidance formula:}
    \[
    \epsilon_\text{guided} = \epsilon_\theta(x_t, t, c_\text{null}) + w \cdot (\epsilon_\theta(x_t, t, c_\text{text}) - \epsilon_\theta(x_t, t, c_\text{null}))
    \]
    where $w$ is the guidance scale.
    \item \textbf{Effect:} Higher $w$ → stronger adherence to text prompt
\end{enumerate}

\textbf{Inference Parameters:}
\begin{itemize}
    \item \textbf{Scheduler:} DDPM with squared cosine schedule
    \item \textbf{Number of inference steps:} 50
    \item \textbf{Random seed:} 422 (for reproducibility)
    \item \textbf{Guidance scales tested:} $w \in \{0, 5, 10, 20, 50, 100\}$
\end{itemize}

\subsubsection{Results}

Figure~\ref{fig:mnist_generation} presents a comprehensive evaluation of the model's generation capabilities across all digit classes and various guidance scales. This systematic comparison demonstrates how the guidance scale parameter $w$ influences both image quality and text-prompt adherence across different digit classes.

\begin{figure}[h]
    \centering
    \includegraphics[width=\textwidth]{../figures/experiments/mnist_generation.png}
    \caption{Comprehensive generation results for all MNIST digit classes (0-9, rows) across different guidance scales ($w \in \{0, 1, 2, 3, 5, 10, 20\}$, columns). Each row shows generations for the prompt "A handwritten digit X" where X corresponds to the digit class. All images generated using 50 inference steps. This visualization demonstrates the model's ability to generate diverse digits with varying degrees of text-prompt adherence controlled by the guidance scale parameter.}
    \label{fig:mnist_generation}
\end{figure}

\textbf{Extended Training Results:}

To evaluate the impact of extended training, we trained the model for 20 epochs and regenerated samples across all digit classes. Figure~\ref{fig:mnist_generation_20epochs} shows the improved generation quality achieved with extended training, demonstrating better convergence and more consistent digit generation across all classes.

\begin{figure}[h]
    \centering
    \includegraphics[width=\textwidth]{../figures/experiments/mnist_generation_20epochs.png}
    \caption{Generation results after 20 epochs of training for all MNIST digit classes (0-9, rows) across different guidance scales ($w \in \{0, 1, 2, 3, 5, 10, 20\}$, columns). All images generated using 50 inference steps. Compared to Figure~\ref{fig:mnist_generation}, the extended training produces sharper digits with improved text-prompt alignment and reduced artifacts, particularly noticeable at higher guidance scales.}
    \label{fig:mnist_generation_20epochs}
\end{figure}

\subsubsection{Guidance-Scale ($w$) Ablation Study}

We systematically evaluated the impact of guidance scale on generation quality:

\textbf{Tested Guidance Scales:}
\begin{itemize}
    \item $w = 0$: Unconditional generation (no text guidance)
    \item $w = 5$: Weak guidance
    \item $w = 10$: Moderate guidance
    \item $w = 20$: Strong guidance
    \item $w = 50$: Very strong guidance
    \item $w = 100$: Maximum guidance
\end{itemize}

\textbf{Evaluation Protocol:}
\begin{enumerate}
    \item Generate images with fixed prompt: "A handwritten digit \{0-9\}"
    \item Use fixed random seed (422) set within the generation function, ensuring consistent initial noise for each guidance scale comparison
    \item Apply 50 denoising steps per image
    \item Qualitatively assess:
    \begin{itemize}
        \item Image clarity and sharpness
        \item Adherence to text prompt (correct digit class)
        \item Presence of artifacts or distortions
        \item Overall sample quality
    \end{itemize}
\end{enumerate}

\textbf{Expected Results:}
\begin{itemize}
    \item $w = 0$: Random digit generation (unconditional)
    \item $w = 5$-$10$: Balanced quality and prompt following
    \item $w = 20$-$50$: Strong prompt adherence with sharp outputs
    \item $w = 100$: Potential over-saturation and artifacts
\end{itemize}

\subsubsection{Visualization}

Generated images displayed using matplotlib with:
\begin{itemize}
    \item Grayscale colormap
    \item Grid layout for guidance scale comparison (1 row $\times$ 6 columns)
    \item Individual subplot titles showing guidance scale values
    \item Tight layout to prevent overlap
\end{itemize}

\subsection{CIFAR-10 Text-to-Image Generation with Classifier-Free Guidance}

\begin{figure}[h]
    \centering
    \includegraphics[width=\textwidth]{../figures/experiments/samples_w10.png}
    \caption{Generated CIFAR-10 images for all 10 classes using guidance scale $w=10$. Each row shows 4 samples for a class: airplane, automobile, bird, cat, deer, dog, frog, horse, ship, and truck. The model demonstrates the ability to generate diverse natural images conditioned on text prompts.}
    \label{fig:cifar10_samples}
\end{figure}

\subsubsection{Objective}
Building upon the success of the MNIST experiment, this experiment extends the text-to-image diffusion framework to CIFAR-10, a more challenging dataset with natural color images. CIFAR-10 contains 32$\times$32 RGB images across 10 object classes, presenting significantly greater complexity than grayscale handwritten digits. This experiment evaluates whether the same architectural principles and classifier-free guidance approach can scale to more realistic image generation tasks.

\subsubsection{Model Architecture}

\textbf{Text Encoder (Frozen):}
\begin{itemize}
    \item \textbf{Model:} CLIP (openai/clip-vit-base-patch32)
    \item \textbf{Embedding dimension:} 512
    \item \textbf{Tokenizer max length:} 77 tokens (standard CLIP length)
    \item \textbf{Training:} Weights are frozen
\end{itemize}

\textbf{Denoising Network (U-Net):}

The U-Net architecture is scaled up compared to the MNIST experiment to handle the increased complexity of natural color images.

\begin{itemize}
    \item \textbf{Architecture:} Custom UNet2DConditionModel for CIFAR-10
    \item \textbf{Input/Output:} 3 channels (RGB), 32$\times$32 pixels
    \item \textbf{Block channels:} (128, 256, 256, 512) --- larger than MNIST to capture natural image features
    \item \textbf{Layers per block:} 2
    \item \textbf{Down blocks:} DownBlock2D $\rightarrow$ CrossAttnDownBlock2D $\rightarrow$ CrossAttnDownBlock2D $\rightarrow$ DownBlock2D
    \item \textbf{Up blocks:} UpBlock2D $\rightarrow$ CrossAttnUpBlock2D $\rightarrow$ CrossAttnUpBlock2D $\rightarrow$ UpBlock2D
    \item \textbf{Cross-attention dimension:} 512 (matches CLIP embedding size)
    \item \textbf{Attention head dimension:} 32
    \item \textbf{Total trainable parameters:} $\sim$45 million (significantly larger than MNIST model)
\end{itemize}

\subsubsection{Dataset}

\textbf{CIFAR-10 Dataset:}
\begin{itemize}
    \item \textbf{Training images:} 50,000
    \item \textbf{Test images:} 10,000
    \item \textbf{Resolution:} 32$\times$32 pixels, RGB color
    \item \textbf{Classes:} 10 object categories:
    \begin{enumerate}
        \setcounter{enumi}{-1}
        \item airplane
        \item automobile
        \item bird
        \item cat
        \item deer
        \item dog
        \item frog
        \item horse
        \item ship
        \item truck
    \end{enumerate}
    \item \textbf{Text captions:} Automatically generated as ``A photo of a \{class\_name\}'' (e.g., ``A photo of a cat'')
    \item \textbf{Preprocessing:} Normalized to [-1, 1] range for diffusion training
\end{itemize}

\subsubsection{Training Configuration}

\begin{itemize}
    \item \textbf{Batch size:} 128 (reduced from MNIST due to larger model and RGB images)
    \item \textbf{Learning rate:} $10^{-4}$
    \item \textbf{Optimizer:} AdamW with weight decay 0.01
    \item \textbf{Epochs:} 50
    \item \textbf{Noise scheduler:} DDPM with linear beta schedule
    \item \textbf{Beta range:} $\beta_{\text{start}} = 0.0001$, $\beta_{\text{end}} = 0.02$
    \item \textbf{Timesteps:} 1,000
    \item \textbf{Loss function:} Mean Squared Error (MSE) between predicted and actual noise
    \item \textbf{Unconditional dropout:} 10\% (for classifier-free guidance training)
\end{itemize}

\textbf{Training Pipeline per Batch:}
\begin{enumerate}
    \item Convert class labels to text captions using CLIP tokenizer
    \item Encode captions to semantic embeddings via frozen CLIP text encoder [batch, 77, 512]
    \item With 10\% probability, replace text embedding with null embedding (empty string) for CFG training
    \item Sample random timestep $t \sim \text{Uniform}(0, 1000)$ for each image
    \item Add Gaussian noise to images: $x_t = \sqrt{\bar{\alpha}_t} x_0 + \sqrt{1-\bar{\alpha}_t} \epsilon$
    \item Predict noise: $\epsilon_\theta(x_t, t, c_{\text{text}})$ using UNet with cross-attention conditioning
    \item Calculate MSE loss: $\mathcal{L} = \|\epsilon - \epsilon_\theta(x_t, t, c_{\text{text}})\|^2$
    \item Backpropagate and update UNet parameters only (CLIP remains frozen)
\end{enumerate}

\subsubsection{Inference Configuration}

\begin{itemize}
    \item \textbf{Scheduler:} DDPM with linear beta schedule
    \item \textbf{Number of inference steps:} 50
    \item \textbf{Guidance scales tested:} $w \in \{0, 2, 5, 10\}$
\end{itemize}

\subsubsection{Evaluation Metrics}

Two complementary metrics were used to evaluate generation quality:

\textbf{1. Fréchet Inception Distance (FID):}
\begin{itemize}
    \item Measures distributional similarity between generated and real images
    \item Computed using pytorch-fid with Inception-v3 features (2048 dimensions)
    \item Lower FID indicates better image quality and diversity
    \item 1,000 generated images compared against 1,000 real CIFAR-10 test images
\end{itemize}

\textbf{2. Classification Accuracy:}
\begin{itemize}
    \item Measures prompt adherence using a pre-trained ResNet-18 classifier
    \item Generated images classified and compared to intended class from prompt
    \item Higher accuracy indicates better text-image alignment
    \item 100 images per class (1,000 total) evaluated per guidance scale
\end{itemize}

\subsubsection{Results}

\begin{table}[h]
    \centering
    \caption{CIFAR-10 Generation Metrics Across Guidance Scales}
    \label{tab:cifar10_metrics}
    \begin{tabular}{|c|c|c|}
        \hline
        \textbf{Guidance Scale ($w$)} & \textbf{FID Score} $\downarrow$ & \textbf{Accuracy (\%)} $\uparrow$ \\
        \hline
        0 (unconditional) & 77.05 & 9.10 \\
        \hline
        2 & \textbf{56.28} & 15.40 \\
        \hline
        5 & 63.13 & 15.00 \\
        \hline
        10 & 77.19 & \textbf{16.50} \\
        \hline
    \end{tabular}
\end{table}

Figure~\ref{fig:cifar10_fid_accuracy} shows the relationship between guidance scale and both metrics, revealing the quality-adherence trade-off characteristic of classifier-free guidance.

\begin{figure}[h]
    \centering
    \includegraphics[width=\textwidth]{../figures/experiments/fid_and_accuracy.png}
    \caption{FID Score (left) and Classification Accuracy (right) vs. Guidance Scale for CIFAR-10 generation. Lower FID indicates better image quality, while higher accuracy indicates better prompt adherence. The optimal guidance scale $w=2$ achieves the best FID (56.28), while $w=10$ achieves the highest accuracy (16.50\%).}
    \label{fig:cifar10_fid_accuracy}
\end{figure}

\textbf{Key Observations:}

\begin{itemize}
    \item \textbf{$w=0$ (Unconditional):} FID of 77.05 with near-random accuracy (9.1\%), confirming that without guidance, the model produces class-agnostic samples.
    
    \item \textbf{$w=2$ (Weak Guidance):} Achieves the \textbf{best FID score of 56.28}, indicating this guidance level produces the most realistic-looking images while maintaining reasonable diversity.
    
    \item \textbf{$w=5$ (Moderate Guidance):} Slight increase in FID to 63.13, suggesting some loss of image quality as the model prioritizes text conditioning.
    
    \item \textbf{$w=10$ (Strong Guidance):} \textbf{Highest accuracy of 16.50\%} but FID degrades to 77.19, demonstrating the classic quality-adherence trade-off.
\end{itemize}

\subsubsection{Quality vs. Adherence Trade-off}

Figure~\ref{fig:cifar10_tradeoff} visualizes the fundamental trade-off between image quality (FID) and text-prompt adherence (accuracy) across guidance scales.

\begin{figure}[h]
    \centering
    \includegraphics[width=0.8\textwidth]{../figures/experiments/quality_vs_adherence.png}
    \caption{Quality vs. Prompt Adherence trade-off curve for CIFAR-10 generation. Each point represents a different guidance scale. The x-axis shows FID (inverted, so rightward is better quality), and the y-axis shows classification accuracy. The curve illustrates that increasing guidance improves prompt adherence at the cost of image quality.}
    \label{fig:cifar10_tradeoff}
\end{figure}

\subsubsection{Per-Class Analysis}

Figure~\ref{fig:cifar10_confusion} shows the confusion matrices for all guidance scales, revealing which classes are most challenging for the model to generate correctly.

\begin{figure}[h]
    \centering
    \includegraphics[width=\textwidth]{../figures/experiments/confusion_matrices.png}
    \caption{Normalized confusion matrices for CIFAR-10 generation across all guidance scales ($w \in \{0, 2, 5, 10\}$). Rows represent the intended class from the text prompt, columns represent the classifier's prediction. Diagonal values indicate correct generation. The matrices reveal that some classes (e.g., truck, ship) are easier to generate correctly than others (e.g., cat, dog).}
    \label{fig:cifar10_confusion}
\end{figure}

\textbf{Class-wise Performance Insights:}
\begin{itemize}
    \item \textbf{Vehicle classes} (airplane, automobile, ship, truck): Generally show higher accuracy, possibly due to more distinct shapes and less intra-class variation.
    \item \textbf{Animal classes} (cat, dog, bird, deer, horse, frog): Show more confusion between similar animals, reflecting the challenge of generating fine-grained visual distinctions.
    \item \textbf{Common misclassifications}: Cat$\leftrightarrow$Dog confusion is prominent, as is Bird$\leftrightarrow$Airplane, likely due to similar silhouettes.
\end{itemize}

\subsubsection{Comparison with MNIST Experiment}

\begin{table}[h]
    \centering
    \caption{Comparison of MNIST and CIFAR-10 Experiments}
    \label{tab:mnist_cifar10_comparison}
    \begin{tabular}{|l|c|c|}
        \hline
        \textbf{Aspect} & \textbf{MNIST} & \textbf{CIFAR-10} \\
        \hline
        Image size & 28$\times$28 & 32$\times$32 \\
        \hline
        Channels & 1 (grayscale) & 3 (RGB) \\
        \hline
        Model parameters & $\sim$3.1M & $\sim$45M \\
        \hline
        Training epochs & 20 & 50 \\
        \hline
        Optimal guidance & $w \in [10, 20]$ & $w = 2$ (quality) / $w = 10$ (adherence) \\
        \hline
        Generation quality & High (simple domain) & Moderate (complex domain) \\
        \hline
    \end{tabular}
\end{table}

\subsubsection{Discussion}

The CIFAR-10 experiment demonstrates that the text-conditioned diffusion framework can scale to more complex natural image domains, though with notable challenges:

\begin{enumerate}
    \item \textbf{Increased Complexity:} Natural images require significantly more model capacity (45M vs 3.1M parameters) and longer training (50 vs 20 epochs).
    
    \item \textbf{Quality-Adherence Trade-off:} Unlike MNIST where higher guidance consistently improved results, CIFAR-10 shows a clear trade-off where the optimal guidance scale depends on whether quality (FID) or adherence (accuracy) is prioritized.
    
    \item \textbf{Classification Limitations:} The moderate accuracy values (9-17\%) are partially attributable to the pre-trained classifier not being fine-tuned on CIFAR-10, and the inherent difficulty of 32$\times$32 classification.
    
    \item \textbf{FID Scores:} FID values around 56-77 indicate reasonable but not state-of-the-art image quality. For comparison, leading CIFAR-10 generative models achieve FID scores below 10.
\end{enumerate}

\subsubsection{Limitations and Future Work}

\begin{itemize}
    \item \textbf{Resolution:} The 32$\times$32 resolution limits the visual detail achievable. Future work could explore upscaling or higher-resolution training.
    
    \item \textbf{Classifier Fine-tuning:} A CIFAR-10-specific classifier fine-tuned on the training set would provide more reliable accuracy metrics.
    
    \item \textbf{Extended Training:} Additional training epochs or larger batch sizes may improve both FID and accuracy.
    
    \item \textbf{Advanced Architectures:} Incorporating techniques from more recent diffusion models (e.g., attention mechanisms, larger models) could improve generation quality.
\end{itemize}

\subsection{WikiArt Text-to-Image Generation with Classifier-Free Guidance}

\subsubsection{Objective}
This experiment extends the text-to-image diffusion framework to the WikiArt dataset, a large-scale collection of fine art paintings spanning 27 distinct artistic styles. Unlike MNIST (28$\times$28 grayscale digits) and CIFAR-10 (32$\times$32 images of 10 classes), WikiArt presents significantly greater challenges: higher resolution (128$\times$128), complex artistic compositions, and subtle stylistic variations that require the model to learn nuanced visual features. The primary objective is to demonstrate that our CLIP-conditioned diffusion approach with classifier-free guidance can scale to higher-resolution, fine-grained artistic generation tasks, matching or exceeding the capabilities demonstrated in related work \cite{yagil2023can}.

The work of Yagil et al.~\cite{yagil2023can} concluded by highlighting the potential of more advanced generative models, such as diffusion models, to further improve the fidelity and quality of generated artistic images. In this experiment, we directly address this open challenge by applying a diffusion-based framework to the WikiArt dataset. Our results empirically validate the hypothesis proposed in their future work section: diffusion models, when combined with powerful text encoders and classifier-free guidance, substantially enhance the realism and diversity of generated art compared to earlier approaches \footnote{Yagil et al.~\cite{yagil2023can} did not report quantitative metrics such as FID or IS, so direct comparison is not possible. In this work, we include such metrics to enable objective evaluation.}.

\subsubsection{Dataset}

\textbf{WikiArt Dataset (HuggingFace: huggan/wikiart):}
\begin{itemize}
    \item \textbf{Training images:} $\sim$81,000 paintings
    \item \textbf{Resolution:} Variable (average original resolution measured across 1000 sampled images: $1640 \times 1627$ pixels, we resized to 128$\times$128 for training)
    \item \textbf{Dataset size on disk:} $\sim$32 GB
    \item \textbf{Storage format:} 72 Apache Parquet files, each containing $\sim$1,100 samples
    \item \textbf{Classes:} 27 art styles:
    \begin{multicols}{3}
    \begin{enumerate}
        \item Abstract Expressionism
        \item Action Painting
        \item Analytical Cubism
        \item Art Nouveau Modern
        \item Baroque
        \item Color Field Painting
        \item Contemporary Realism
        \item Cubism
        \item Early Renaissance
        \item Expressionism
        \item Fauvism
        \item High Renaissance
        \item Impressionism
        \item Mannerism Late Renaissance
        \item Minimalism
        \item Naive Art Primitivism
        \item New Realism
        \item Northern Renaissance
        \item Pointillism
        \item Pop Art
        \item Post Impressionism
        \item Realism
        \item Rococo
        \item Romanticism
        \item Symbolism
        \item Synthetic Cubism
        \item Ukiyo-e
    \end{enumerate}
    \end{multicols}
    \item \textbf{Text captions:} Automatically generated as ``A painting in the style of \{style\_name\}''
    \item \textbf{Preprocessing:} Resized to 128$\times$128, normalized to [-1, 1] range
\end{itemize}

\subsubsection{Dataset Loading Challenge and Solution}
\label{sec:wikiart-dataloader}

A significant engineering challenge emerged when loading the WikiArt dataset for training. The standard approach of using PyTorch's \texttt{DataLoader} with \texttt{shuffle=True} proved impractical for this large-scale dataset stored across multiple Parquet files \footnote{Parquet is a columnn data storage format that efficiently compresses and stores structured data.}.

\textbf{The Problem:}

The WikiArt dataset is stored as 72 Parquet files:
\begin{verbatim}
train-00000-of-00072.parquet
train-00001-of-00072.parquet
...
train-00071-of-00072.parquet
\end{verbatim}

Each file is approximately 400--450 MB and contains $\sim$1,100 image samples with embedded image bytes. When using a standard PyTorch \texttt{Dataset} with random shuffling, the \texttt{DataLoader} requests samples by random global indices. This causes severe performance degradation:

\begin{enumerate}
    \item \textbf{Random access pattern:} A batch of 16 images might require loading 16 different Parquet files
    \item \textbf{Repeated I/O:} Each file (400 MB) must be read from disk for every sample
    \item \textbf{Result:} Batch loading times of 10--15 seconds (instead of $<$1 second)
\end{enumerate}

Initial attempts to mitigate this with LRU caching of loaded files led to memory exhaustion when combined with multi-process \texttt{DataLoader} workers, as each worker maintains its own cache.

\textbf{The Solution: File-Sequential Iterable Dataset}

We implemented a custom \texttt{IterableDataset} that processes files sequentially:

\begin{enumerate}
    \item \textbf{File-level iteration:} Load one Parquet file entirely into memory ($\sim$400 MB)
    \item \textbf{Per-file shuffling:} Shuffle the $\sim$1,100 samples within the loaded file
    \item \textbf{Yield all samples:} Return all samples from this file before moving to the next
    \item \textbf{File order shuffling:} Randomize file order at the start of each epoch
    \item \textbf{Memory release:} Free memory after exhausting each file
\end{enumerate}

\begin{figure}[htbp]
\centering
\small
\begin{verbatim}
Epoch: Shuffled File Order
      ↓
    File 0      File 1      File 2     ...    File 71
    (1100)      (1100)      (1100)           (1100)
      ↓           ↓           ↓                  ↓
  Shuffle     Shuffle     Shuffle          Shuffle
   samples     samples     samples          samples
      ↓           ↓           ↓                  ↓
   Yield all   Yield all   Yield all       Yield all
   samples     samples     samples         samples
      ↓           ↓           ↓                  ↓
  Release      Release      Release        Release
  memory       memory       memory         memory
\end{verbatim}
\caption{File-sequential loading strategy for WikiArt dataset. Each epoch processes files in shuffled order, with per-file sample shuffling and memory release after processing.}
\label{fig:wikiart-loading}
\end{figure}

\textbf{Performance Comparison:}

\begin{table}[htbp]
\centering
\begin{tabular}{lcc}
\toprule
\textbf{Loading Strategy} & \textbf{Batch Time} & \textbf{Memory Usage} \\
\midrule
Random access (naive) & 10--15 sec & Variable (cache misses) \\
LRU cache (5 files) & 3--5 sec & $\sim$2 GB \\
File-sequential (ours) & 0.1--0.5 sec & $\sim$400 MB \\
\bottomrule
\end{tabular}
\caption{Batch loading performance for WikiArt dataset (batch size 16).}
\label{tab:wikiart-loading-perf}
\end{table}

\textbf{Trade-off Analysis:}

The file-sequential approach sacrifices global shuffling for practical training speed. Instead of shuffling across all 81,000 samples, shuffling occurs within groups of $\sim$1,100 samples (one file). This trade-off is acceptable because:

\begin{itemize}
    \item File order is randomized each epoch, providing inter-epoch diversity
    \item Similar approaches are used in production systems (WebDataset, TensorFlow Datasets)
    \item Training convergence was not noticeably affected in practice
    \item The 20--100$\times$ speedup enables practical iteration during development
\end{itemize}

\subsubsection{Model Architecture}

\textbf{Text Encoder (Frozen):}
\begin{itemize}
    \item \textbf{Model:} CLIP (openai/clip-vit-base-patch32)
    \item \textbf{Embedding dimension:} 512
    \item \textbf{Tokenizer max length:} 77 tokens
    \item \textbf{Training:} Weights are frozen
\end{itemize}

\textbf{Denoising Network (U-Net):}

The U-Net architecture is scaled up significantly compared to previous experiments to handle the increased resolution and complexity of artistic images.

\begin{itemize}
    \item \textbf{Architecture:} Custom UNet2DConditionModel for WikiArt
    \item \textbf{Input/Output:} 3 channels (RGB), 128$\times$128 pixels
    \item \textbf{Block channels:} (128, 256, 512, 512, 1024) --- 5 resolution levels for 128$\times$128
    \item \textbf{Layers per block:} 2
    \item \textbf{Down blocks:} DownBlock2D $\rightarrow$ CrossAttnDownBlock2D $\rightarrow$ CrossAttnDownBlock2D $\rightarrow$ CrossAttnDownBlock2D $\rightarrow$ DownBlock2D
    \item \textbf{Up blocks:} UpBlock2D $\rightarrow$ CrossAttnUpBlock2D $\rightarrow$ CrossAttnUpBlock2D $\rightarrow$ CrossAttnUpBlock2D $\rightarrow$ UpBlock2D
    \item \textbf{Cross-attention dimension:} 512 (matches CLIP embedding size)
    \item \textbf{Attention head dimension:} 32
    \item \textbf{Total trainable parameters:} $\sim$150 million
\end{itemize}

\subsubsection{Training Configuration}

\begin{itemize}
    \item \textbf{Batch size:} 64 (optimized after performance profiling, see \S\ref{sec:wikiart-optimization})
    \item \textbf{Learning rate:} $10^{-5}$ (smaller than CIFAR-10 due to larger model)
    \item \textbf{Optimizer:} AdamW with weight decay 0.01
    \item \textbf{Epochs:} 100
    \item \textbf{Noise scheduler:} DDPM with linear beta schedule
    \item \textbf{Beta range:} $\beta_{\text{start}} = 0.0001$, $\beta_{\text{end}} = 0.02$
    \item \textbf{Timesteps:} 1,000
    \item \textbf{Loss function:} Mean Squared Error (MSE) between predicted and actual noise
    \item \textbf{Unconditional dropout:} 10\% (for classifier-free guidance training)
    \item \textbf{Checkpoint frequency:} Every 10 epochs
\end{itemize}

\subsubsection{Training Pipeline}

\begin{enumerate}
    \item Load batch of images and style labels from file-sequential iterator
    \item Convert style labels to text captions (e.g., ``A painting in the style of Impressionism'')
    \item Encode captions using frozen CLIP text encoder
    \item Apply 10\% unconditional dropout (replace embeddings with null embedding)
    \item Sample random noise and timesteps
    \item Add noise to images according to DDPM schedule
    \item Predict noise using U-Net conditioned on text embeddings
    \item Compute MSE loss and update weights
\end{enumerate}

\subsubsection{Training Performance Optimization}
\label{sec:wikiart-optimization}

To ensure efficient use of computational resources, we conducted a detailed performance profiling of the training pipeline. The analysis revealed a critical bottleneck that significantly impacted training throughput.

\textbf{Initial Performance Analysis:}

We instrumented the training loop to measure the time spent in each stage of a single training iteration with an initial batch size of 16. The breakdown revealed:

\begin{table}[htbp]
\centering
\begin{tabular}{lrr}
\toprule
\textbf{Operation} & \textbf{Time (ms)} & \textbf{Percentage} \\
\midrule
Data loading & 1857.2 & 70.4\% \\
Move to device & 0.5 & 0.0\% \\
Text encoding & 78.8 & 3.0\% \\
CFG masking & 9.0 & 0.3\% \\
Noise addition & 8.3 & 0.3\% \\
Forward pass & 272.6 & 10.3\% \\
Loss calculation & 7.2 & 0.3\% \\
Backward pass & 405.9 & 15.4\% \\
\midrule
\textbf{Total} & \textbf{2639.5} & \textbf{100.0\%} \\
\bottomrule
\end{tabular}
\caption{Training pipeline performance breakdown with batch size 16. Data loading dominates at 70.4\% of total iteration time.}
\label{tab:wikiart-bottleneck}
\end{table}

\textbf{Bottleneck Identification:}

The profiling revealed that \emph{data loading} consumed 70.4\% of the training time (1857.2 ms per batch), while the actual model computation (forward + backward pass) accounted for only 25.7\% (678.5 ms). This indicates that the GPU was severely underutilized, spending most of its time idle while waiting for data.

\textbf{Root Cause:}

Despite the efficient file-sequential loading strategy (\S\ref{sec:wikiart-dataloader}), the small batch size of 16 created a mismatch between I/O throughput and GPU compute capacity. The A100 80GB GPU used for training can process 128$\times$128 RGB images at significantly higher throughput than batch size 16 provides.

\textbf{Solution: Batch Size Scaling}

We conducted GPU memory profiling to determine the maximum feasible batch size:

\begin{enumerate}
    \item Measured peak GPU memory usage with batch size 16: $\sim$8 GB / 80 GB (10\% utilization)
    \item Selected batch size 64 (power of 2) for optimal GPU memory alignment and stability
\end{enumerate}

\textbf{Performance Impact:}

Increasing the batch size from 16 to 64 provided a 4$\times$ increase in throughput:

\begin{table}[htbp]
\centering
\begin{tabular}{lcccc}
\toprule
\textbf{Batch Size} & \textbf{Time/Batch} & \textbf{Batches/Epoch} & \textbf{Time/Epoch} & \textbf{Speedup} \\
\midrule
16 & 2.64 sec & 5,063 & 3.7 hours & 1.0$\times$ \\
64 & 2.8 sec & 1,266 & 0.99 hours & 3.7$\times$ \\
\bottomrule
\end{tabular}
\caption{Training performance comparison before and after batch size optimization. The larger batch size reduces training time per epoch from 3.7 hours to 59 minutes.}
\label{tab:wikiart-batch-size-optimization}
\end{table}

This optimization reduced the total training time for 100 epochs from approximately 15.4 days to 4.1 days, making the experiment practical to execute on available hardware.

\subsubsection{Comparison Across Experiments}

Table~\ref{tab:experiment-comparison} summarizes the key differences across all three experiments.

\begin{table}[htbp]
\centering
\small
\begin{tabular}{lccc}
\toprule
\textbf{Aspect} & \textbf{MNIST} & \textbf{CIFAR-10} & \textbf{WikiArt} \\
\midrule
Resolution & 28$\times$28 & 32$\times$32 & 128$\times$128 \\
Channels & 1 (grayscale) & 3 (RGB) & 3 (RGB) \\
Classes & 10 digits & 10 objects & 27 styles \\
Training samples & 60,000 & 50,000 & $\sim$81,000 \\
Dataset size & $\sim$50 MB & $\sim$170 MB & $\sim$32 GB \\
U-Net blocks & 4 & 4 & 5 \\
Parameters & $\sim$25M & $\sim$45M & $\sim$150M \\
Batch size & 256 & 128 & 64 \\
Learning rate & $10^{-4}$ & $10^{-4}$ & $10^{-5}$ \\
Epochs & 20 & 50 & 100 \\
\bottomrule
\end{tabular}
\caption{Comparison of experimental configurations across MNIST, CIFAR-10, and WikiArt datasets.}
\label{tab:experiment-comparison}
\end{table}

% Results section to be added after training completes
% \subsubsection{Results}
% TODO: Add training curves, generated samples, and FID scores after training
